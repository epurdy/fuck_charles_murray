\documentclass[11pt]{article}

\title{Fuck Charles Murray: Transcending the IQ Paradigm (DRAFT)}

\author{Eric Purdy}

\date{\today}

\usepackage{amsmath}
\usepackage{amsthm}
\usepackage{amssymb}

\usepackage{fancyhdr}
\usepackage{epigraph}

\pagestyle{fancy}
\fancyhf{} % clear all header and footer fields
\fancyhead[C]{Draft} % Centered header on every page
\fancyfoot[C]{Draft} % Centered footer on every page
\renewcommand{\headrulewidth}{0.4pt} % header rule width
\renewcommand{\footrulewidth}{0.4pt} % footer rule width

\setlength{\epigraphwidth}{.8\textwidth}
\renewcommand{\epigraphflush}{flushright}
\renewcommand{\epigraphrule}{0pt}

\newtheorem{law}{Law}
\newtheorem{theorem}{Theorem}
\newtheorem{conjecture}{Conjecture}

\begin{document}

\maketitle

\epigraph{Them what has, gets.}{-- Folk saying}


\begin{abstract}
We describe the bones of a scientific paradigm to replace the IQ
paradigm.
\end{abstract}

\section{Introduction}

The observed correlation between IQ and outcomes is an empirical fact
that must be explained away before we can throw out the IQ paradigm.

We posit the following chain of causation:

\begin{enumerate}
\item The rich give their children fancy educations, basically
  booksmarts.
\item The poor give their children the educations they can afford,
  basically streetsmarts and trade skills.
\item IQ tests predominantly measure the former, since the rich people
  control the academic establishment that certifies them
\item IQ tests are administered by people who don't know the answers
  and have personal prejudice, making them just a very fancy and
  subtle/deniable version of poll tests with a grandfather clause.
\item Outcomes are biased towards the rich because capital usually
  wins ultimatum games over labor (Not always, though. Solidarity!)
\end{enumerate}

The first loose thread that proves this paradigm is Richard
Feynman. Born to a poor Jewish family, he had all the fancy Jewish
wisdom and all the fancy trade skills and streetsmarts, but probably
lacked some of the book learning that the test was measuring, and
maybe his IQ test administrator hated Jews, or disliked Feynman upon
meeting him, or had an argument with his wife that morning, etc. His
measured IQ was surprisingly low, given that he ended up being one of
the great geniuses of human history. (He also failed a sanity test
when he was about to get drafted, if I recall correctly from his
books.)

The second loose thread that proves this paradigm is that black people
are really good at chess, far better than some of the most celebrated
geniuses who have walked the Earth. I personally beat Paul Erd\H{o}s
in a game of chess at the age of 12 or 13. I took a few chess lessons
from Ramdake Lewis, a black chess master from Cincinnati, around the
same time, and learned a few really good tricks. If IQ existed, and
was rank-ordered in any way, and correlated to race, this would be a
devastatingly unlikely series of events: my IQ at the time would have
had to have been higher than Erdos's and lower than Lewis's, meaning
that Lewis would have had to have been one of the smartest people to
ever live. Maybe he was and is (I haven't kept up with him). But
certainly he's not as famous or as celebrated as Paul Erd\H{o}s.

The third loose thread that proves the paradigm is the existence of
``Jewish questions'' in the Russian academic system. These were
incredibly tricky math problems that were given only and specifically
to Jewish applicants to academic opportunities. Ultimately, some
Jewish people figured out how to crack this style of puzzle and made
sure to spread the knowledge within the community. This particular
phenomenon, together with the primacy of math in academia, is probably
sufficient to explain Jewish overrepresentation in math and the
sciences.

The fourth loose thread that proves the paradigm is Albert Einstein:
why the hell was he working as a patent clerk instead of being at a
German university? The answer is blindingly obvious: the antisemitism
of the German academic establishment meant that no Jewish person could
ever be smart enough to reliably earn their place, not even Einstein.

There is thus no reason to presume that race has anything to do with
IQ, since IQ is a meaningless construct. To the extent that there are
measured differences between races, it really just reflects the match
between the education received by the test designer and that received
by the test subject.

\section{Strengths-based Intelligence Modeling}

Consider that each person has some number of skill statistics. Each
skill statistic represents their skill in some particular human
endeavor. It need not even be an intellectual endeavor, it could be
basketball or football or hockey or soccer or chess or whatever.

Model each skill statistic using either the Elo or Glicko rating
system in competitive games. (Preferably the Glicko rating system,
since it is more empirically grounded and stable.) This rating can
either be a unipartite rating in competitive games or a bipartite
rating in feats of strength, like the chess puzzles on Chess.com.
This rating can exist either for individual in individual
competitions, or teams in team competitions.

That's it. That's the framework.

\section{TODO}

\begin{enumerate}
\item Make a D\&D style campaign manual that illustrates the above
  framework, and invite people to play it.

\end{enumerate}

\end{document}
